\documentclass[11pt, oneside]{article}   	% use "amsart" instead of "article" for AMSLaTeX format
\usepackage{geometry}                		% See geometry.pdf to learn the layout options. There are lots.
\geometry{letterpaper}                   		% ... or a4paper or a5paper or ... 
%\geometry{landscape}                		% Activate for rotated page geometry
%\usepackage[parfill]{parskip}    		% Activate to begin paragraphs with an empty line rather than an indent
\usepackage{graphicx}				% Use pdf, png, jpg, or eps§ with pdflatex; use eps in DVI mode
								% TeX will automatically convert eps --> pdf in pdflatex		
\usepackage{amssymb}
\usepackage{amsmath}
\usepackage{dcolumn}
\usepackage[table, svgnames, x11names]{xcolor}

\addtolength{\oddsidemargin}{-.875in}
	\addtolength{\evensidemargin}{-.875in}
	\addtolength{\textwidth}{1.75in}

	\addtolength{\topmargin}{-.875in}
	\addtolength{\textheight}{1.75in}



\newcolumntype{d}[1]{D{.}{.}{#1}}

%SetFonts

%SetFonts


\title{Computational modeling}
\author{Nicolas P. Cottaris}
\date{}							% Activate to display a given date or no date

\begin{document}
\maketitle
\section{ISETBio computational model}
\subsection{Overview}
To connect the measured RGC spatial transfer functions (STF) to the underlying retinal anatomy, a computational model is employed which simulates optical, spectral, spatial, and temporal components of the AO stimulation apparatus, as well as the monkey's optics and cone mosaic structure. The model computes an STF, assuming that cone signals are pooled by the center and surround mechanisms of an RGC according to a difference of Gaussians (DoG) spatial profile model. The parameters of the DoG model and therefore, the actual cone pooling weights, are estimating by minimizing the error between the predicted and measured STFs. A schematic overview of this model is depicted in Figure \ref{fig:ModelOverview}. Various model parameters used to simulate various aspects of the experimental conditions are listed in Table \ref{table:ModelParameters}.

\subsection{AO-delivered stimulus modeling}
The AO-delivered drifting monochromatic sinusoidal gratings used to measured RGC STFs during the experiment, are modeled as temporal sequences of ISETBio spatial spectral radiance scenes, where each scene models a different frame of the drifting stimulus. The spectral profile of the monochromatic beam, the spatial extent and resolution, and the temporal characteristic of the AO display subsystem as are all taken into account in generating these scenes. 

\subsection{Retinal stimulus modeling}
The generated scenes are subsequently passed via a diffraction-limited optical system model which accounts for blur in optical in a perfect AO system and which can also account for additional residual amount blur that may occur in a slightly imperfect AO system, for example due to a slight defocus of the stimulus with respect to the plane of cone inner segments. (Say something more here??). The amount of residual blur in not known a-priori, and it is estimated from the model as described below. 

\subsection{Cone excitation response modeling}
In the next stage, an ISETBio model of the monkey's cone mosaic is generated using an iterative approach [Cottaris et al] from cone density maps measured during AO imaging. The temporal photon absorption excitation response, $E^j(\omega,t)$, of each cone-$j$ of the mosaic, to a drifting grating of spatial frequency $\omega$, is computed from the corresponding temporal sequence of retinal spatial spectral irradiance images, after weighing retinal irradiance by each cone's spectral quantal efficiency and spatially integrating within each cone's inner segment aperture. 

\subsection{Converting cone excitation responses to cone contrast responses}
Assuming that cones are adapted to the mean background irradiance, the excitation response of a cone-$j$ to a stimulus with spatial frequency $\omega$, $E^j(\omega,t)$, is converted to contrast response, $R^j(\omega, t)$, by first subtracting the excitation to the background stimulus, $E^j_o$, and then dividing by it, separately for each cone-$j$, i.e.:
\begin{equation}
R^j(\omega, t) = \frac{E^j(\omega,t) - E^j_o}{E^j_o}
\end{equation}
\noindent where
$E^j_o$ is the excitation of cone-$j$, to the zero contrast (background) stimulus. This operation mimics the photocurrent generation process which converts cone absorption events in the inner segment into ionic currents flowing through the cone outer segment, and which in effect normalizes stimulus-induced excitations responses with respect to the background cone excitation responses.



\subsection{Computing model ganglion cells responses}
Model ganglion cells responses, $\mbox{RGC}(\omega,t)$, are computed from the cone contrast responses assuming linear spatial pooling of cone contrast signals by the antagonistic center and the surround mechanisms, as follows:
\begin{eqnarray}
\mbox{RGC}(\omega,t) & = & \mbox{RGC}_c(\omega,t) - \mbox{RGC}_s(\omega,t) \\
& = & \sum_{j} W_{c}^j  \times R^j(\omega, t) -  \sum_{j} W_{s}^j  \times R^j(\omega, t)
\end{eqnarray}
%
\noindent where, $W_{c}^j, W_{s}^j$, are the spatial pooling weights with which the center and surround, respectively, mechanisms are summing the $R^j(\omega, t)$ responses. There is no temporal filtering, and there is no delay introduced in the computation of center, $\mbox{RGC}_c(\omega,t)$, and surround $\mbox{RGC}_s(\omega,t)$ responses. 

\subsection{Computing model ganglion cell spatial transfer functions}
The spatial transfer function (STF) of a model RGC, $\mbox{STF}^{m}(\omega)$, is computed as: 
\begin{equation}
\mbox{STF}^{m}(\omega) = \mbox{STF}_{c}^{m}(\omega) - \mbox{STF}_{s}^{m}(\omega)
\end{equation}
%
\noindent with
%
$\mbox{STF}_{c}^{m}(\omega)$ and $\mbox{STF}_{s}^{m}(\omega)$ derived from the amplitudes of the best fit sinusoidal functions to $\mbox{RGC}_c(\omega,t)$, and $\mbox{RGC}_s(\omega,t)$, respectively, where the sinusoid frequency is fixed to the temporal frequency of the drifting gratings, and its phase is allowed to vary. Formulating the model STF as the difference of the center and the surround STFs, enables the model STF to achieve negative values, which are some times encountered in the measured fluorescence-based STF data for the lowest spatial frequencies.

\subsection{Estimating cone pooling weights from the fluorescence-based STF measurements}

The weights with which the center and surround mechanisms of a model RGC, are pooling signals from a cone-$j$ are computed as follows:
\begin{eqnarray}
W_c^j  & = & \begin{cases}
   k_c \times \exp \left [ -\left( d_{j}/r_c \right) ^2 \right ], & \text{for the multi-cone RF center model}.\\
   k_c, & \text{for the single-cone RF center model}.
   \end{cases} \\
W_s^j &= &k_s \times \exp \left [ -\left( d_{j}/r_s \right) ^2 \right ]
\end{eqnarray}

\noindent where $d_j$ is the distance between cone-$j$ and the spatial position of the center mechanism of the model RGC. The parameters $k_c$, $k_s$, $r_c$ and $r_s$, which represent the center and surround peak sensitivities and characteristic radii, respectively, of the Difference of Gaussians (DoG) RF model, are determined by minimizing the weighted error between the model STF, $\mbox{STF}^{m}(\omega)$, and the measured STF, $\mbox{STF}^{\Delta F / F}(\omega)$, accumulated over all spatial frequencies, $\omega$:

\begin{equation}
\mbox{RMSE} = \displaystyle \sum_{\omega} \frac{1}{\epsilon({\omega})} {\left [  \mbox{STF}^{m}(\omega)  - \mbox{STF}^{\Delta F / F}(\omega) \right ] }^2
\end{equation}
where $\epsilon({\omega})$ is the standard error of the mean of the $\mbox{STF}^{\Delta F / F}(\omega)$ measurement. To minimize the chance of getting stuck to local minima of the error function, we employ a multi-start minimizer which is ran 256 times, keeping the results from the run which results in the minimum RMSE.


\subsection{Model validation \& model selection}
To validate the extracted RF model parameters ($k_c$, $k_s$, $r_c$ and $r_s$), we employ a cross-validation approach, in which we train the model by minimizing the error between $\mbox{STF}^{m}$ and the fluorescence-based STF, $\mbox{STF}^{\Delta F / F}_{s_{train}}$, measured at one recording session, $s_{train}$, while assessing model performance by computing the error between $\mbox{STF}^{m}_{s_{train}}$ and $\mbox{STF}^{\Delta F / F}_{s_{test}}$, with $s_{test} \ne  s_{train}$. In this, cross-session comparison, we allow for an arbitrary scaling of $\mbox{STF}^{m}_{s_{train}}$ by a factor $\gamma$, which is selected so as to minimize the error between model-- and  fluorescence--based STF, i.e., 

\begin{equation}
\mbox{RMSE}^{CV, s_{test}}_{s_{train}} = \sum_{\omega} \left [ \gamma \times \mbox{STF}^{m}_{s_{train}}(\omega) - \mbox{STF}^{\Delta F / F}_{s_{test}} \right ] ^2
\end{equation}

Averaging $\mbox{RMSE}^{CV, s_{test}}_{s_{train}}$ over all test sessions, we obtain the overall cross-validated RMSE for each training session, $\mbox{RMSE}^{CV}_{s_{train}}$, and repeating this over all training sessions, we determine the training session with the minimal $\mbox{RMSE}^{CV}_{s_{train}}$, say $s_{train}^{best}$. The model computed at the $s_{train}^{best}$ session is considered to be the best generalizing (over unseen data) model.


Since the receptive field location of the recorded RGCs is only known approximately to lie within the central 40 $\mu m$, we compute $\mbox{RMSE}^{CV}_{s_{train}^{best}}$ a number of times, each time assuming that the model RGC receives its center-driving signal from cones in different parts of the central 40 $\mu m$ of the model cone mosaic, and we find the position with the minimal $\mbox{RMSE}^{CV}_{s_{train}^{best}}$. The corresponding model, is the best-generalizing model with the lowest cross-validation error amongst all assumed RF center positions, and its $k_c$, $k_s$, $r_c$ and $r_s$ values are the best estimates of the DoG model for that cell.

This cross-validation approach can also be used to select between models with different number of parameters, for example the single-cone RF center model, where the $r_c$ parameter is absent (in effect fixed to the underlying cone characteristic radius), vs. the multi-cone RF center model, where the $r_c$ parameter is allowed to vary, and to also assess the residual defocus in the AO apparratus.

\newpage


\begin{table}% put at top of page if possible 
\centering
\begin{tabular}{|r d{3.3}|}
%\begin{tabular}{|r l|}
\hline
\rowcolor{LightSlateGray!35!Lavender} \multicolumn{2}{|l|}{\textbf{retinal modeling (optics)}} \\
\hline
\mbox{pupil diameter} ($mm$) : & 6.7  \\
\mbox{retinal magnification factor} ($\mu m \times deg^{-1}$) : & 199.26 \\
\hline
\hline
\rowcolor{LightSlateGray!35!Lavender} \multicolumn{2}{|l|}{\textbf{retinal modeling (cone mosaic)}} \\
\hline
\mbox{size} (degs) : & \multicolumn{1}{c|}{1.3 $\times$ 1.3}\\
\mbox{max. density} ($10^3 \mbox{cones} \times mm^{-2}$) : & 270.20\\  
\mbox{cone aperture profile} : & \multicolumn{1}{c|}{\mbox{Gaussian}}\\
\mbox{cone aperture characteristic radius} : & \multicolumn{1}{c|}{0.204 $\times \sqrt{2} \times$
\mbox{i.s. diam.}}\\
\mbox{foveal cone characteristic radius (arc.min.)} : & \multicolumn{1}{c|}{0.17}\\
\mbox{L-cone ratio} : & \multicolumn{1}{c|}{0.48}\\
\mbox{M-cone ratio} : & \multicolumn{1}{c|}{0.48}\\
\mbox{S-cone ratio} : & \multicolumn{1}{c|}{0.04}\\
\hline
\hline
\rowcolor{LightSlateGray!35!Lavender}  \multicolumn{2}{|l|}{\textbf{visual stimulation modeling}} \\
\hline
\mbox{monochromatic stimulation (peak)} ($nm$) : & 561.0  \\
\mbox{monochromatic stimulation (FWHM)} ($nm$) : & 5.0  \\
\mbox{retinal pixel size} ($\mu m$) : & 1.03  \\
\mbox{drift rate} ($Hz$) : & 6.0  \\
\mbox{refresh rate} ($Hz$) : & 25.3  \\
\mbox{spatial extent} ($degs$) : & \multicolumn{1}{c|}{0.7 $\times$ 0.7}\\
\mbox{mean power} ($mW \times cm^2$) : & 1.29  \\
\mbox{contrast} : & 1.0 \\
\hline
\end{tabular}
\caption{Modeling parameters}\label{table:ModelParameters}
\end{table}


\begin{figure}[htbp] %  figure placement: here, top, bottom, or page
   \centering
   \includegraphics[width=7in]{Figures/ModelOverview.pdf} 
   \caption{Schematic overview of the ISETBio computational model.}
   \label{fig:ModelOverview}
\end{figure}



\end{document}  